\section{Unit Tests}
\begin{itemize}
    \item Consider following Test Driven Development rules
    \begin{enumerate}
        \item ``You may not write production code until you have written a failing unit test.''
        \item ``You may not write more of a unit test than is sufficient to fail, and not compiling is failing.''
        \item ``You may not write more production code than is sufficient to pass the currently failing test.''
    \end{enumerate}
    \item Keep tests clean
    \begin{itemize}
        \item Having dirty tests is equivalent to, if not worse than, having no tests
        \item Unit tests keep production code flexible, maintainable and reusable
        \item Tests don't need to be as efficient as production code
    \end{itemize}
    \item Use the Build-Operate-Check pattern for each test method
    \begin{itemize}
        \item Build up test data and test environment
        \item Execute the test
        \item Verify results
    \end{itemize}
    \item Keep the number of asserts small
    \begin{itemize}
        \item May require building up a testing language (given, when, then)
        \item Consider testing a single concept per test
    \end{itemize}
    \item Follow the F.I.R.S.T principle
    \begin{itemize}
        \item Fast: if tests are slow, they will be executed less frequently
        \item Independent: tests should not depend on each other
        \item Repeatable: tests should run offline and on every environment
        \item Self-Validating: tests should be validated by the system (not manually)
        \item Timely: write tests just before the production code and not after
    \end{itemize}
\end{itemize}
