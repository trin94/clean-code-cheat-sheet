\section{Formatting}
\begin{itemize}
    \item Agree to a formatting standard
    \begin{itemize}
        \item Improves readability and maintainability
        \item Even more important in team environments
        \item The style survives, the code does not
    \end{itemize}
    \item Apply the newspaper metaphor
    \begin{itemize}
        \item The name of the class/module should be sufficient to tell if that's the correct source file
        \item At the top: high level and important information
        \item At the bottom: details increase
        \item Most articles (read: source files) are small, some are a bit longer
    \end{itemize}
    \item Divide concepts with blank lines
    \begin{itemize}
        \item Add blank lines between imports, classes and functions
    \end{itemize}
    \item Keep an eye on vertical density
    \begin{itemize}
        \item Avoid blank lines between variables that are part of the same concept
    \end{itemize}
    \item Keep an eye on vertical distance
    \begin{itemize}
        \item Concepts that are closely related should be kept close to each other
        \item Closely related concepts should not be separated into different files
        \item Therefore avoid protected variables
    \end{itemize}
    \item Declare variables at the right location
    \begin{itemize}
        \item Local variables should be declared as close to their usage as possible
        \item In short functions: at the top
    \end{itemize}
    \item Keep dependent functions together
    \begin{itemize}
        \item The caller should be above the callee
    \end{itemize}
\end{itemize}
