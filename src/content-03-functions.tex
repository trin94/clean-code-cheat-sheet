\section{Functions}
\begin{itemize}
    \item Make functions small
    \begin{itemize}
        \item They should hardly ever be 20 lines long
    \end{itemize}
    \item Use descriptive names
    \begin{itemize}
        \item The smaller and focused a function, the easier it is to choose a descriptive name
        \item Long descriptive name $>$ long descriptive comment
    \end{itemize}
    \item Check blocks and indenting
    \begin{itemize}
        \item Bodies of \textit{if}, \textit{else} and \textit{while} should be one line long:
        a function call with a nicely descriptive name
        \item Indent level should not be greater than one or two
    \end{itemize}
    \item Do one thing
    \begin{itemize}
        \item \texttt{Functions should do one thing.}
        \item [] \texttt{They should do it well.}
        \item [] \texttt{They should do it only.}
        \item If the function does only those steps that are one level below the stated name of the function,
        then the function is doing one thing
        \item Functions should either do something or answer something, but not both
    \end{itemize}
    \item Implement one level of abstraction per function
    \begin{itemize}
        \item Helps to differentiate better between concept and detail
    \end{itemize}
    \item Follow the Stepdown Rule
    \begin{itemize}
        \item Should read like a newspaper
        \item High level functions at the top, details at the bottom
    \end{itemize}
    \item Use switch statements and if/else chains rarely
    \begin{itemize}
        \item They do more than one thing by nature
        \item There aren't many reasons to use switch statements in high level classes
        \item Often a sign of missed polymorphism
    \end{itemize}
    \newpage
    \item Be careful with function arguments
    \begin{itemize}
        \item $0$ arguments (niladic) $>1$ argument (monadic) $>2$ arguments (dyadic) $>3$ arguments (triadic)
        \item Never use more than $3$ arguments (polyadic)
        \item Less arguments means easier testing and better readability
    \end{itemize}
    \item Avoid output arguments
    \begin{itemize}
        \item Interrupts the reading flow
        \item \textit{StringBuilder transform(StringBuilder in)} $>$ \textit{void transform(StringBuilder out)}
    \end{itemize}
    \item Avoid flag arguments
    \begin{itemize}
        \item Clear sign of a function that does more than one thing
    \end{itemize}
    \item Dyadic functions
    \begin{itemize}
        \item Perfectly reasonable if arguments have a natural cohesion or a natural ordering
    \end{itemize}
    \item Triadic functions
    \begin{itemize}
        \item Difficult to read but sometimes required
        \item Maybe better to wrap arguments into an \textit{Argument Object}
    \end{itemize}
    \item Variable arguments
    \begin{itemize}
        \item If variable arguments are treated identically, they are equivalent to a single argument
    \end{itemize}
    \item Avoid side effects
    \begin{itemize}
        \item Clear sign that the function does more than one thing
        \item Function does \textit{hidden} things
    \end{itemize}
    \item Do not follow the "One entry, one exit" rule
    \begin{itemize}
        \item Only beneficial in large functions, which we want to avoid
        \item Having multiple \textit{return}, \textit{break} or \textit{continue} statements may be more expressive
    \end{itemize}
    \item Avoid duplicate code
\end{itemize}
